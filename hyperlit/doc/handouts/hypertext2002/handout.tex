%
% Modification History
%
% 2002-June-6   Jason Rohrer
% Created.
%


\documentclass[12pt]{article}
\usepackage{amssymb,latexsym,fullpage,epsfig,pstricks}


\begin{document}

\newcommand{\http}{{\sc http}}
\newcommand{\url}{{\sc url}}


% syntax: %\insertfigure{width}{file_name}{caption}{label}
\newcommand{\insertfigure}[4] {
	\begin{figure}[tb]
		\begin{center}
			\fbox{
				\begin{minipage}{#1}
					\includegraphics[width=\textwidth]{#2}
				
					\caption{#3}
				
					\label{#4}
				\end{minipage}
			} 
		\end{center}
	\end{figure}
}

\newlength{\oldUnitLength}

% syntax: %\insertpicture{width}{height_ratio}{caption}{label}{picture_commands}
\newcommand{\insertpicture}[5] {
	\begin{figure}[tb]
		\begin{center}
			\fbox{
				\begin{minipage}{#1}
                    \begin{pspicture*}(0,0)(\textwidth, #2\textwidth)
                    \psset{xunit=\textwidth}
                    \psset{yunit=\textwidth}
                    \psset{runit=\textwidth}                 
                        #5
                    \end{pspicture*}
					\caption{#3}
				
					\label{#4}
				\end{minipage}
			} 
		\end{center}
	\end{figure}
}






\author{Jason Rohrer [rohrer@cse.ucsc.edu]}
\date{ACM Hypertext Conference 2002}
\title{hyperlit\\ http://hypertext.sourceforge.net/hyperlit}


\maketitle


\begin{abstract}
What kind of monolithic hypertext system can be created in a single night of coding?
\end{abstract}


\section{What is it?}
hyperlit is a small Java application for authoring literary hypertext.


\section{What is it not?}
hyperlit is similar (in a very weak sense) to Storyspace.  However, it is not as powerful as Storyspace in that it does not provide spatial views of a hypertext.  Then again, Storyspace costs \$295 and has been developed over many years, while hyperlit is free and was developed in a single night.

\section{What can it do?}
hyperlit is expressive enough to write substantial literary hypertext, as long as the final product does not require a spatial view.  For example, one could use it to author a work as substantial as ``afternoon, a story'', the flagship hypertext for Storyspace.

\section{What else can it do?}
hyperlit can export hypertext lexia as a network of HTML pages for the web.

\section{Where is an example?}
``six and a half seconds, a story'' was authored with hyperlit and has been exported for the web.  To read this story, go to
\begin{center}
{\tt http://jasonrohrer.n3.net}
\end{center}

\section{What do I need to run it?}
hyperlit requires Java 1.2 or later.

\section{Who is responsible?}
hyperlit was coded in a single night by Jason Rohrer.

\insertfigure{4in}{hyperlit.eps}{The hyperlit user interface.}{fig:screen}


\end{document}

